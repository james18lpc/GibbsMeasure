% In this file you should put the actual content of the blueprint.
% It will be used both by the web and the print version.
% It should *not* include the \begin{document}
%
% If you want to split the blueprint content into several files then
% the current file can be a simple sequence of \input. Otherwise It
% can start with a \section or \chapter for instance.

\chapter{Intro}

\begin{definition}[Cylinder Events]
    \label{def:cylinder-event}
    \lean{cylinder-event}
    \leanok
    Let $(E,\mathcal{E})$ be a measurable space, and let $S$ be a set. Then,
    \begin{align*}
        \mathcal{F}:\mathcal{P}(S)&\to\mathcal{P}(E)\\
        \Delta&\mapsto \sigma(\{\text{proj}_\delta:E^S\to E\mid \delta\in \Delta\})
    \end{align*}
    defines the \textbf{cylinder events in }$\Delta$ (for each $\Delta\in\mathcal{P}(S)$), where each $\text{proj}_\delta$ is precisely the canonical projection map.
\end{definition}

\begin{definition}[Probability Kernel]
    \label{def:kernel}
    \lean{ProbabilityTheory.kernel}
    \leanok
    Let $(X,\mathcal{X})$ and $(Y,\mathcal{Y})$ be measurable spaces. Then,
    \begin{equation*}
        \text{ProbKer}_{\mathcal{Y},\mathcal{X}}:=\left\{\pi:\mathcal{X}\times Y\to\overline{\mathbb{R}_{\geq0}}~\middle\vert~ \forall y\in Y,\pi(\cdot\mid y)\in\mathfrak{P}(X,\mathcal{X});~\forall A\in\mathcal{X},\pi(A\mid\cdot)\text{ is }\mathcal{Y}\text{-measurable}\right\}
    \end{equation*}
    defines the set of \textbf{probability kernels from $\mathcal{Y}$ to $\mathcal{X}$}, where $\mathfrak{P}(X,\mathcal{X})$ is the space of probability measures on $X$.
\end{definition}

\begin{definition}[Specification]
    \label{def:specification}
    \lean{Specification}
    \leanok
    \uses{def:cylinder-event, def:kernel}
    Let $(E,\mathcal{E})$ be a measurable space, and let $S$ be a set. Then,
    \begin{equation*}
        \Gamma_{\mathcal{E},S}:=\left\{\gamma\in\prod_{\Lambda\in\text{Finset}(S)}\text{ProbKer}_{\mathcal{F}_{S\setminus\Lambda},\mathcal{E}^S}~\middle\vert~ \forall \Lambda,\Delta\in\text{Finset}(S),\Lambda\subseteq\Delta\implies\gamma_\Lambda\circ_k\gamma_\Delta=\gamma_\Delta\right\}
    \end{equation*}
    defines the set of \textbf{specifications with parameter set $S$ and state space $(E,\mathcal{E})$}, i.e. it is the set of indexed collections of probability kernels that satisfy a consistency criterion.
\end{definition}

\begin{definition}[Gibbs Measures]
    \label{def:gibbs-measure}
    \lean{GibbsMeasure}

    \uses{def:specification}
    Let $(E,\mathcal{E})$ be a measurable space, and let $S$ be a set. Then,
    \begin{align*}
        \mathcal{G}:\Gamma_{\mathcal{E},S}&\to\mathcal{P}(\mathfrak{P}(E^S,\mathcal{E}^S))\\
        \gamma&\mapsto\left\{\mu\in\mathfrak{P}(E^S,\mathcal{E}^S)~\middle\vert~\mu(A\mid\mathcal{F}_{S\setminus\Lambda})=\gamma_\Lambda(A\mid\cdot)~\mu\text{-a.s. for each } A\in\mathcal{F}_S\text{ and }\Lambda\in\text{Finset}(S)\right\}
    \end{align*}
    defines the set of \textbf{Gibbs measures specified by $\gamma$} (for each $\gamma\in\Gamma_{\mathcal{E},S}$).
\end{definition}

\begin{lemma}[Characterization of Gibbs Measures]
    \label{lem:gibbs-measure-char}
    \lean{ProbabilityTheory.cdf}

    \uses{def:gibbs-measure}
    Let $(E,\mathcal{E})$ be a measurable space, and let $S$ be a set. Let $\gamma$ be a \textbf{proper} specification with parameter set $S$ and state space $(E,\mathcal{E})$, and let $\mu\in\mathfrak{P}(E^S,\mathcal{F}_S)$. TFAE:
    \begin{enumerate}
        \item $\mu\in\mathcal{G}(\gamma)$.
        \item $\gamma_\Lambda\circ_k\mu=\mu$ for all $\Lambda\in\text{Finset}(S)$.
        \item There is a cofinal subset $\mathcal{S}\subseteq\text{Finset}(S)$ such that $\gamma_\Lambda\circ_k\mu=\mu$ for all $\Lambda\in\mathcal{S}$.
    \end{enumerate}
\end{lemma}

%\begin{definition}[Product Probability Measure]
%    \label{def:productProbabilityMeasure}
%    \lean{ProbabilityTheory.cdf}
%    characteristic predicate?
%\end{definition}

\begin{definition}[Superposition]
    \label{def:superposition}
    \lean{extend}
    \leanok
    Let $E$ and $S$ be sets. Let $\Delta\in\mathcal{P}(S)$, and let $\omega\in E^S$. We define
    \begin{align*}
        \text{super}_\omega:E^\Delta&\to E^S\\
        \zeta&\mapsto \delta\mapsto\begin{cases}
            \zeta_\delta & \delta\in\Delta\\
            \omega_\delta & \delta\notin\Delta
        \end{cases}
    \end{align*}
    to be the \textbf{superposition of $\zeta$ on $\omega$} (for each $\zeta\in E^\Delta$).
\end{definition}

\begin{definition}[Independent Specification with Single Spin Distribution [ISSSD]]
    \label{def:ISSSD}
    \lean{isssd}
    \uses{def:superposition}
    Let $(E,\mathcal{E})$ be a measurable space, and let $S$ be a set. Then,
    \begin{align*}
        \text{ISSSD}:\mathfrak{P}(E,\mathcal{E})&\to\text{Finset}(S)\to\mathcal{E}^S\times E^S\to\overline{\mathbb{R}_{\geq0}}\\
        \nu&\mapsto\Lambda\mapsto(A\mid\omega)\mapsto\left(\nu^\Lambda\left(\text{super}_\omega^{-1}(A)\right)\right)
    \end{align*}
    defines the \textbf{Independent Specification with Single Spin Distribution with $\nu$} (for each $\nu\in\mathfrak{P}(E,\mathcal{E})$), where $\nu^\Lambda$ is the usual product measure.
\end{definition}

\begin{lemma}[ISSSDs are Specifications]
    \label{lem:isssd-specification}
    \lean{}
    \uses{def:ISSSD, def:specification}
    The image of $\text{ISSSD}$ is always a collection of specifications.
\end{lemma}

\begin{lemma}[Strong Consistency of ISSSDs]
    \label{lem:isssd-strong-consistency}
    \lean{}
    \uses{lem:isssd-specification}
    Let $(E,\mathcal{E})$ be a measurable space, and let $S$ be a set. Then, $\text{ISSSD}$ satisfies $\text{ISSSD}(\mu)_\Delta\circ_k\text{ISSSD}(\mu)_\Lambda=\text{ISSSD}(\mu)_{\Delta\cup\Lambda}$ for all $\Delta,\Lambda\in\text{Finset}(S)$ and $\mu\in\mathfrak{P}(E,\mathcal{E})$.
\end{lemma}

\begin{lemma}[Existence of Gibbs Measures Specified by ISSSDs]
    \label{lem:gibbs-measure-existence}
    \lean{}
    \uses{def:gibbs-measure, lem:isssd-specification}
    Let $(E,\mathcal{E})$ be a measurable space, and let $S$ be a set. Then, $\mathcal{G}(\text{ISSSD}(\mu))=\{\mu^S\}$.
\end{lemma}
