\chapter{Prereqs}

\begin{definition}[Cylinder Events]
    \label{def:cylinder-event}
    \lean{cylinderEvents}
    \leanok

    Let $(E,\mathcal{E})$ be a measurable space, and let $S$ be a set. Then,
    \begin{align*}
        \mathcal{F}:\mathcal{P}(S)&\to\mathcal{P}(E)\\
        \Delta&\mapsto \sigma(\{\text{proj}_\delta:E^S\to E\mid \delta\in \Delta\})
    \end{align*}
    defines the \textbf{cylinder events in }$\Delta$ (for each $\Delta\in\mathcal{P}(S)$), where each $\text{proj}_\delta$ is precisely the canonical projection map.
\end{definition}

\begin{definition}[Kernel]
    \label{def:kernel}
    \lean{ProbabilityTheory.Kernel}
    \leanok

    Let $(X,\mathcal{X})$ and $(Y,\mathcal{Y})$ be measurable spaces. Then,
    \begin{equation*}
        \text{Ker}_{\mathcal{Y},\mathcal{X}}:=\left\{\pi:\mathcal{X}\times Y\to\overline{\mathbb{R}}_{\geq0}~\middle\vert~ \forall y\in Y,\pi(\cdot\mid y)\in\mathfrak{M}(X,\mathcal{X});~\forall A\in\mathcal{X},\pi(A\mid\cdot)\text{ is }\mathcal{Y}\text{-measurable}\right\}
    \end{equation*}
    defines the set of \textbf{kernels from $\mathcal{Y}$ to $\mathcal{X}$}, where $\mathfrak{M}(X,\mathcal{X})$ is the space of measures on $X$.

    We say that $\pi\in\text{Ker}_{\mathcal{Y},\mathcal{X}}$ is a \textbf{probability} or \textbf{Markov kernel} iff $\pi(X\mid \cdot)=1$.
\end{definition}

\begin{definition}[Markov Kernel]
    \label{def:markov-kernel}
    \lean{ProbabilityTheory.Kernel}
    \leanok

    Let $(X,\mathcal{X})$ and $(Y,\mathcal{Y})$ be measurable spaces. We say that $\pi\in\text{Ker}_{\mathcal{Y},\mathcal{X}}$ is a \textbf{probability} or \textbf{Markov kernel} iff $\pi(X\mid \cdot)=1$.
\end{definition}

\begin{definition}[Proper Kernel]
    \label{def:proper-kernel}
    \uses{def:kernel}
    \lean{ProbabilityTheory.Kernel.IsProper}
    \leanok

    Let $(X,\mathcal{X})$ be a measurable space, and let $\mathcal{B}$ be a sub $\sigma$-algebra of $\mathcal{X}$. Then, $\pi\in\text{Ker}_{\mathcal{B},\mathcal{X}}$ is \textbf{proper} iff $\pi(A\cap B\mid x)=\pi(A\mid x)\cdot\mathbf{1}_B(x)$ for all $A\in\mathcal{X}$, $B\in\mathcal{B}$ and $x\in X$.
\end{definition}

\begin{definition}[Conditional Expectation Kernel]
    \label{def:cond-exp-kernel}
    \uses{def:kernel}
    \lean{ProbabilityTheory.Kernel.Condexp}
    \leanok

    Let $(X,\mathcal{X})$ be a measurable space, let $\mathcal{B}$ be a sub $\sigma$-algebra of $\mathcal{X}$ and let $\mu\in\mathfrak{M}(X,\mathcal{X})$. Then, $\pi\in\text{Ker}_{\mathcal{B},\mathcal{X}}$ is a \textbf{conditional expectation kernel} iff $\mu(A\mid \mathcal{B})=\pi(A\mid\cdot)$ $\mu$-a.e..
\end{definition}

\begin{definition}[Product Probability Measure]
    \label{def:product-probability-measure}
    \lean{MeasureTheory.productMeasure}
    \leanok

    Let $I$ be a set. Suppose for each $i\in I$ that $(\Omega_i,\mathcal{B}_i,P_i)$ is a probability space. Then, $P\coloneqq\bigotimes_{i\in I}P_i$ is a well-defined product probability measure on $\prod_{i\in I}\Omega_i$.
\end{definition}

\begin{definition}[Juxtaposition]
    \label{def:juxtaposition}
    \lean{juxt}
    \leanok

    Let $E$ and $S$ be sets. Let $\Delta\in\mathcal{P}(S)$, and let $\omega\in E^S$. We define
    \begin{align*}
        \text{super}_\omega:E^\Delta&\to E^S\\
        \zeta&\mapsto \delta\mapsto\begin{cases}
            \zeta_\delta & \delta\in\Delta\\
            \omega_\delta & \delta\notin\Delta
        \end{cases}
    \end{align*}
    to be the \textbf{juxtaposition of $\zeta$ and $\omega$} (for each $\zeta\in E^\Delta$).
\end{definition}
