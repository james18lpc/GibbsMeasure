\section{Prescribing conditional probabilities}

\begin{definition}[Specification]
    \label{def:spec}
    \uses{def:cylinder-event, def:kernel}
    \lean{Specification}
    \leanok{}

    A {\bf specification} is a family of kernels $\gamma : \Finset S \to \Ker_{\mathcal{F}_{S\setminus\Lambda}, \mathcal{E}^S}$ which is {\bf consistent}, in the sense that
    \[\forall \Lambda_1, \Lambda_2 \in \Finset(S), \Lambda_1 \subseteq \Lambda_2 \implies \gamma_{\Lambda_1} \circ_k \gamma_{\Lambda_2} = \gamma_{\Lambda_2}\]
\end{definition}

All specifications will be with parameter set $S$ and state space $(E, \mathcal{E})$ in this chapter.

\begin{definition}[Independent specification]
    \label{def:indep-spec}
    \uses{def:spec}
    \lean{Specification.IsIndep}
    \leanok{}

    A specification $\gamma$ is \textbf{independent} iff
    \[\forall \Lambda_1, \Lambda_2 \in \Finset(S), \gamma_{\Lambda_1} \circ_k \gamma_{\Lambda_2} = \gamma_{\Lambda_1\cup\Lambda_2}\]
\end{definition}

\begin{definition}[Markov specification]
    \label{def:markov-spec}
    \lean{Specification.IsMarkov}
    \leanok{}
    \uses{def:spec}

    A specification $\gamma$ is a \textbf{Markov specification} iff $\gamma_\Lambda$ is a probability kernel for every $\Lambda \in \Finset(S)$.
\end{definition}

\begin{definition}[Proper specification]
    \label{def:proper-spec}
    \uses{def:spec, def:proper-ker}
    \lean{Specification.IsProper}
    \leanok{}

    A specification $\gamma$ is \textbf{proper} iff the kernel $\gamma_\Lambda$ is proper for every $\Lambda \in \Finset(S)$.
\end{definition}

\begin{definition}[Gibbs measures]
    \label{def:gibbs-meas}
    \uses{def:spec, def:cond-exp-ker}
    \lean{Specification.IsGibbsMeasure}
    \leanok
    Given a specification $\gamma$, a \textbf{Gibbs measures specified by $\gamma$} is a measure $\nu \in \mathfrak{M}(E^S, \mathcal{E}^S)$ such that $\gamma_\Lambda(A|\cdot)$ is a conditional expectation kernel for $\nu$ for all $A \in \mathcal E^S$ and $\Lambda \in \Finset(S)$.
\end{definition}

\begin{lemma}[Characterisation of Gibbs measures, Remark 1.24]
    \label{lem:gibbs-meas-tfae}
    \uses{def:gibbs-meas, def:proper-spec}
    \lean{Specification.isGibbsMeasure_iff_forall_bind_eq, Specification.isGibbsMeasure_iff_frequently_bind_eq}
    \leanok

    Let $\gamma$ be a {\it proper} specification with parameter set $S$ and state space $(E, \mathcal{E})$, and let $\nu\in\mathfrak{P}(E^S, \mathcal{E}^S)$. TFAE:
    \begin{enumerate}
        \item $\nu\in\mathcal{G}(\gamma)$.
        \item $\gamma_\Lambda\circ_m\nu = \nu$ for all $\Lambda\in\Finset(S)$.
        \item $\gamma_\Lambda\circ_m\nu = \nu$ frequently as $\Lambda \to S$.
    \end{enumerate}
\end{lemma}
\begin{proof}
    \uses{lem:cond-exp-proper-ker-lintegral}
    \leanok

    1 is equivalent to 2 by Lemma \ref{lem:cond-exp-proper-ker-lintegral}. 2 trivially implies 3. Now, 3 implies 2 because for each $\Lambda$ there exists some $\Lambda' \supseteq \Lambda$ such that
    $\gamma_{\Lambda'}\circ_k\nu = \nu$. Then
    $$\nu\gamma_\Lambda = \nu\gamma_{\Lambda'}\gamma_\Lambda = \nu\gamma_{\Lambda'} = \nu$$
\end{proof}
