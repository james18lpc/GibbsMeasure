\section{Preliminaries}

\begin{definition}[Juxtaposition]
    \label{def:juxtaposition}
    \lean{juxt}
    \leanok

    Let $E$ and $S$ be sets. Let $\Delta\in\mathcal{P}(S)$, and let $\omega\in E^S$. We define
    \begin{align}
        \juxtBy{\omega} : E^\Delta&\to E^S\\
        \zeta&\mapsto \delta\mapsto\begin{cases}
            \zeta_\delta & \delta\in\Delta\\
            \omega_\delta & \delta\notin\Delta
        \end{cases}
    \end{align}
    to be the \textbf{juxtaposition of $\zeta$ and $\omega$} (for each $\zeta\in E^\Delta$).
\end{definition}

\begin{definition}[Cylinder events]
    \label{def:cylinder-event}
    \lean{cylinderEvents}
    \leanok

    Let $(E,\mathcal{E})$ be a measurable space, and let $S$ be a set. Then,
    \begin{align}
        \mathcal{F}:\mathcal{P}(S)&\to \set{\text{sigma algebras on } E^S}\\
        \Delta&\mapsto \sigma(\{\text{proj}_\delta:E^S\to E\mid \delta\in \Delta\})
    \end{align}
    defines the \textbf{cylinder events in }$\Delta$ (for each $\Delta\in\mathcal{P}(S)$),
    where each $\text{proj}_\delta$ is the coordinate projection at coordinate $\delta$.
\end{definition}

\begin{definition}[Kernel]
    \label{def:kernel}
    \lean{ProbabilityTheory.Kernel}
    \leanok

    Let $(X,\mathcal{X})$ and $(Y,\mathcal{Y})$ be measurable spaces. Then,
    \begin{equation*}
        \text{Ker}_{\mathcal{Y},\mathcal{X}}:=\left\{\pi:\mathcal{X}\times Y\to\overline{\mathbb{R}}_{\geq0}~\middle\vert~ \forall y\in Y,\pi(\cdot\mid y)\in\mathfrak{M}(X,\mathcal{X});~\forall A\in\mathcal{X},\pi(A\mid\cdot)\text{ is }\mathcal{Y}\text{-measurable}\right\}
    \end{equation*}
    defines the set of \textbf{kernels from $\mathcal{Y}$ to $\mathcal{X}$}, where $\mathfrak{M}(X,\mathcal{X})$ is the space of measures on $X$.

    %We say that $\pi\in\text{Ker}_{\mathcal{Y},\mathcal{X}}$ is a \textbf{probability} or \textbf{Markov kernel} iff $\pi(X\mid \cdot)=1$.
\end{definition}

\begin{definition}[Markov kernel]
    \label{def:markov-ker}
    \uses{def:kernel}
    \lean{ProbabilityTheory.IsMarkovKernel}
    \leanok

    Let $(X,\mathcal{X})$ and $(Y,\mathcal{Y})$ be measurable spaces. We say that $\pi\in\text{Ker}_{\mathcal{Y},\mathcal{X}}$ is a \textbf{probability} or \textbf{Markov kernel} iff $\pi(X\mid \cdot)=1$.
\end{definition}

Let $(X, \mathcal X)$ be a measurable space, let $\mathcal B$ be a sub $\sigma$-algebra of $\mathcal X$. Let $\pi \in \text{Ker}_{\mathcal B, \mathcal X}$.

\begin{definition}[Proper kernel]
    \label{def:proper-kernel}
    \uses{def:kernel}
    \lean{ProbabilityTheory.Kernel.IsProper}
    \leanok

    A kernel $\pi\in\text{Ker}_{\mathcal{B},\mathcal{X}}$ is \textbf{proper} iff $\pi(A\cap B\mid x)=\pi(A\mid x)\cdot\mathbf{1}_B(x)$ for all $A\in\mathcal{X}$, $B\in\mathcal{B}$ and $x\in X$.
\end{definition}

\begin{lemma}[Integral characterisation of proper kernels]
    \label{lem:proper-kernel-integral}
    \uses{def:proper-kernel}
    \lean{ProbabilityTheory.Kernel.IsProper.lintegral_mul, ProbabilityTheory.Kernel.IsProper.integral_mul}
    \leanok

    If kernel $\pi\in\text{Ker}_{\mathcal{B},\mathcal{X}}$ is proper, then
    $$\int_x f(x) g(x)\ d\pi(\cdot\mid x_0) = g(x_0)\int_x f(x)\ d\pi(\cdot\mid x_0)$$
    for all $A \in \mathcal X$, $B \in \mathcal B$, $x_0 \in X$ and functions
    \begin{enumerate}
        \item $f, g : X \to [0, \infty[$ such that $f$ is $\mathcal X$-measurable, $g$ is $\mathcal B$-measurable.
        \item $f, g : X \to \R$ such that $f$ is $\mathcal X$-integrable, $g$ is $\mathcal B$-integrable.
    \end{enumerate}
\end{lemma}
\begin{proof}
    % \uses{}
    % \leanok

    Standard machine.
\end{proof}

\begin{definition}[Conditional expectation kernel]
    \label{def:cond-exp-ker}
    \uses{def:kernel}
    \lean{ProbabilityTheory.condexpKernel, ProbabilityTheory.Kernel.IsCondExp}
    \leanok

    let $\mu\in\mathfrak{M}(X,\mathcal{X})$. Then, $\pi\in\text{Ker}_{\mathcal{B},\mathcal{X}}$ is a \textbf{conditional expectation kernel with respect to $\mu$} iff $\mu(A\mid \mathcal{B})=\pi(A\mid\cdot)$ $\mu$-a.e..
\end{definition}

\begin{lemma}[Characterisation of conditional expectation kernels]
    \label{lem:cond-exp-ker-char}
    \uses{def:cond-exp-ker}
    % \lean{}
    % \leanok

    Let $\mu\in\mathfrak{M}(X,\mathcal{X})$ be a finite measure and let $\pi\in\text{Ker}_{\mathcal{B},\mathcal{X}}$ be a proper Markov kernel. Then,
    $$\pi \text{ is a conditional expectation kernel for }\mu \iff \mu\pi = \mu$$
\end{lemma}

\begin{lemma}[Equivalent definition for Markov conditional expectation kernels]
    \label{lem:cond-exp-ker-markov-char}
    \uses{def:cond-exp-ker, def:markov-ker}
    \leanok

    blach
\end{lemma}

% Remark 1.20
\begin{lemma}[Characterisation of proper conditional expectation kernels]
    \label{lem:cond-exp-ker-proper-tfae}
    \uses{def:cond-exp-ker, def:proper-kernel}
    \lean{ProbabilityTheory.Kernel.isCondExp_iff_bind_eq_left}
    \leanok

    Let $\mu \in \mathfrak{M}(X, \mathcal X)$ be a finite measure and let $\pi\in\text{Ker}_{\mathcal{B},\mathcal{X}}$ be a proper Markov kernel. TFAE:
    \begin{enumerate}
        \item $\pi$ is a conditional expectation kernel for $\mu$.
        \item something
        \item $\mu\circ_m\pi=\mu$.
    \end{enumerate}
\end{lemma}
\begin{proof}
    % \uses{}
    \leanok

    By the characterisation of conditional expectation,
    $$\pi \text{ is a conditional expectation kernel for }\mu \iff \forall A \in \mathcal X, \forall B \in \mathcal B, \mu(A \inter B) = \int_B \pi(A|\cdot)\ \partial\mu$$
    By properness of $\pi$,
    $$\int_B \pi(A|\cdot)\ \partial\mu = \mu\pi(A \inter B)$$
    Hence
    \begin{align}
        \pi \text{ is a cond. exp. kernel with respect to }\mu
        & \iff \forall A \in \mathcal X, \forall B \in \mathcal B, \mu(A \inter B) = \mu\pi(A \inter B) \\
        & \iff \forall A \in \mathcal X, \mu(A) = \mu\pi(A) \\
        & \iff \mu = \mu\pi
    \end{align}
\end{proof}
