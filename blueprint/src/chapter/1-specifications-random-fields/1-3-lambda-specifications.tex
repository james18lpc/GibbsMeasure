\section{\texorpdfstring{$\lambda$}{Lambda}-specifications}

Let $S$ be a set, $(E, \mathcal{E})$ be a measurable space and $\nu$ a measure on $E$.

\begin{definition}[Product probability measure]
    \label{def:product-probability-measure}
    \lean{MeasureTheory.productMeasure}
    \leanok{}

    Let $I$ be a set. Suppose for each $i\in I$ that $(\Omega_i,\MCB_i,P_i)$ is a probability space. Then, $P:=\bigotimes_{i\in I}P_i$ is a well-defined product probability measure on $\prod_{i\in I}\Omega_i$.
\end{definition}

\begin{definition}[Independent Specification with Single Spin Distribution (ISSSD)]
    \label{def:isssd}
    \uses{def:juxtaposition}
    \lean{Specification.isssdFun}
    % \leanok

    The \textbf{Independent Specification with Single Spin Distribution $\nu$} is
    \begin{align}
        \ISSSD:\mathfrak{P}(E, \mathcal{E})&\to\Finset(S)\to\mathcal{E}^S\times E^S\to\overline{\mathbb{R}_{\geq0}}\\
        \nu&\mapsto\Lambda\mapsto(A\mid\omega)\mapsto\left(\nu^\Lambda\left(\juxtBy{\omega}^{-1}(A)\right)\right)
    \end{align}
    defines the \textbf{Independent Specification with Single Spin Distribution with $\nu$} (for each $\nu\in\mathfrak{P}(E, \mathcal{E})$), where $\nu^\Lambda$ is the usual product measure.
\end{definition}

\begin{lemma}[Independence of ISSSDs]
    \label{lem:isssd-isIndep}
    \uses{def:isssd, def:indep-spec}
    \lean{Specification.isssdFun_comp_isssdFun}
    \leanok{}

    $\ISSSD(\nu)$ is independent.
\end{lemma}
\begin{proof}
    % \leanok

  Immediate.
\end{proof}


%TODO: change strong consistency to independent
\begin{definition}[ISSSDs are specifications]
    \label{def:isssd-spec}
    \uses{lem:isssd-strong-consistency, def:spec}
    \lean{Specification.isssd}
    \leanok

    $\ISSSD(\nu)$ is a specification.
\end{definition}

\begin{lemma}[ISSSDs are proper specifications]
    \label{lem:isssd-proper-spec}
    \uses{def:isssd, def:proper-spec}
    \lean{Specification.IsProper.isssd}
    \leanok

    $\ISSSD(\nu)$ is a proper specification.
\end{lemma}
\begin{proof}
    \uses{def:isssd-spec}
    % \leanok

    We already know it's a specification. Properness is immediate.
\end{proof}

\begin{lemma}[Uniqueness of a Gibbs measure specified by an ISSSD]
    \label{lem:isssd-gibbs-meas-uniqueness}
    \uses{def:gibbs-meas, def:isssd-spec}

    There is at most one Gibbs measure specified by $\ISSSD(\nu)$.
\end{lemma}
\begin{proof}
    \uses{lem:isssd-proper-spec}
    % \leanok

    See book.
\end{proof}

\begin{lemma}[Existence of a Gibbs measure specified by an ISSSD]
    \label{lem:isssd-gibbs-meas-existence}
    \uses{def:gibbs-meas, def:product-probability-measure, def:isssd-spec}
    \lean{Specification.isGibbsMeasure_isssd_productMeasure}
    \leanok{}

    The product measure $\nu^S$ is a Gibbs measure specified by $\ISSSD(\nu)$.
\end{lemma}
\begin{proof}
    % \leanok

   Immediate.
\end{proof}

\begin{definition}[Modifier]
    \label{def:modif}
    \uses{def:spec}
    \lean{Specification.IsModifier, Specification.modification}
    \leanok{}

    A {\bf modifier of $\gamma$} is a family
    \[\rho : \Finset(S) \to \Omega \to [0, \infty[\]
    such that the corresponding family of kernels $\rho\gamma$ is a specification.
\end{definition}

\begin{lemma}[Modifier of a modifier]
    \label{lem:modif-modif}
    \uses{def:modif}
    \lean{Specification.modification_modification}
    \leanok{}

    Modifying a specification $\gamma$ by $\rho_1$ then $\rho_2$ is the same as modifying it by their product.
\end{lemma}
\begin{proof}
    \uses{}
    \leanok

    TODO
\end{proof}

% Currently skipping the definition of positive modifiers because it looks a bit dumb

% Start of Remark 1.28.1
\begin{lemma}[A modifier of a proper specification is proper]
    \label{lem:modif-proper}
    \uses{def:modif, def:proper-spec}
    \lean{Specification.IsProper.modification}
    \leanok{}

    If $\gamma$ is a specification and $\rho$ a modifier of $\gamma$, then $\rho\gamma$ is a proper specification.
\end{lemma}
\begin{proof}
    \uses{lem:proper-ker-lintegral}
    \leanok

    For all $\Lambda \in \Finset(S)$, $A \in \mathcal E^S$, $B \in \mathcal{F}_{S\setminus\Lambda}$, $\eta : S \to E$, we want to prove
    $$(\rho\gamma)_\Lambda(A \inter B | \eta) = 1_B(\eta) (\rho\gamma)_\Lambda(A B | \eta)$$
    Expanding out, this is equivalent to
    $$\int_{\zeta \in A \inter B} \rho_\Lambda(\zeta)\ d(\gamma_\Lambda(\eta)) = 1_B(\eta) \int_{\zeta \in A} \rho_\Lambda(\zeta)\ d(\gamma_\Lambda(\eta))$$
    which is true by Lemma \ref{lem:proper-ker-lintegral} with $f = 1_A\rho_\Lambda$, $g = 1_B$.
\end{proof}

% The rest of Remark 1.28.1 is vacuous in our setup

% (Yaël) Don't understand Remark 1.28.2. Can it be generalised away from ISSSD?

% (Yaël) Not sure how to generalise Remark 1.28.3 away from ISSSD

% We can't state Remark 1.28.4 since we don't have a characteristic for a specification to be a
% modifier of another measure

% We don't need this. This is just for fun.
\begin{lemma}[Every specification is a modification of some ISSSD, Remark 1.28.5]
    \label{lem:exists-modif-isssd}
    \uses{def:modif, def:isssd}
    % \lean{}
    % \leanok

    If $E$ is countable, $\nu$ is the counting measure and $\gamma$ is any specification, then
    $$\rho_\Lambda(\eta) = \gamma_\Lambda(\{\sigma_\Lambda = \eta_\Lambda\} | \eta)$$
    is a modifier from $\ISSSD(\nu)$ to $\gamma$.
\end{lemma}
\begin{proof}
    % \uses{}
    % \leanok

    For all $\Lambda \in \Finset(S)$, $A$ measurable, $\eta : S \to E$, we have
    \begin{align}
        (\rho\ISSSD(\nu))_\Lambda(A|\eta)
        & = \int_\zeta \rho_\Lambda(\zeta) \ISSSD(\nu)(d\zeta|\eta) \\
        & = \int_\zeta \gamma_\Lambda(\{\sigma_\Lambda = \eta_\Lambda\} | \eta) \ISSSD(\nu)(d\zeta|\eta) \\
        & = \gamma_\Lambda(A|\eta)
    \end{align}
\end{proof}

\begin{proposition}[Characterisation of modifiers, Proposition 1.30.1]
    \label{prop:modif-char}
    \uses{def:modif}
    \lean{Specification.isModifier_iff_ae_eq}
    \leanok{}

    If $\rho$ is a family of measurable densities and $\gamma$ is a proper specification, then TFAE
    \begin{enumerate}
        \item $\rho$ is a modifier of $\gamma$
        \item For all $\Lambda_1, \Lambda_2$ with $\Lambda_1 \subseteq \Lambda_2$ and all $\eta : S \to E$, we have
        \[\rho_{\Lambda_2} = \rho_{\Lambda_1}\cdot (\gamma_{\Lambda_1} \rho_{\Lambda_2}) \quad \gamma_{\Lambda_2}(\cdot|\eta)\text{-a.e.}\]
    \end{enumerate}
\end{proposition}
\begin{proof}
    % \uses{}
    % \leanok

    \begin{itemize}
        \item ($\implies$) $\rho_{\Lambda_2} = \rho_{\Lambda_1}\cdot (\gamma_{\Lambda_1} \rho_{\Lambda_2}) \quad \gamma_{\Lambda_2}(\cdot|\eta)\text{-a.e.}$
        \begin{itemize}
            \item $\implies \rho_{\Lambda_2}\gamma_{\Lambda_2} = $
        \end{itemize}
    \end{itemize}

\end{proof}

\begin{proposition}[Characterisation of modifiers of independent specifications, Proposition 1.30.2]
    \label{prop:modif-indep-char}
    \uses{def:modif, def:indep-spec}
    \lean{Specification.isModifier_iff_ae_comm}
    \leanok{}

    If $\rho$ is a family of measurable densities and $\gamma$ is a proper independent specification, then TFAE
    \begin{enumerate}
        \item $\rho$ is a modifier of $\gamma$
        \item For all $\Lambda_1, \Lambda_2$ with $\Lambda_1 \subseteq \Lambda_2$, $\eta : S \to E$ and $\gamma_{\Lambda_2 \setminus \Lambda_1}(\cdot|\alpha)$-almost all $\eta_2 : S \to E$, we have
        \[\rho_{\Lambda_2}(\zeta_1)\rho_{\Lambda_1}(\zeta_2) = \rho_{\Lambda_2}(\zeta_2) \rho_{\Lambda_1}(\zeta_1)\]
        for $\gamma_{\Lambda_1}(\cdot|\eta_2) \times \gamma_{\Lambda_2}(\cdot|\eta_2)$-almost all $(\zeta_1, \zeta_2)$.
    \end{enumerate}
\end{proposition}
\begin{proof}
    % \uses{}
    % \leanok

\end{proof}

\begin{definition}[Premodifier, Definition 1.31]
    \label{def:premodif}
    \uses{}
    \lean{Specification.IsPremodifier}
    \leanok

    A family of measurable functions $h_\Lambda : (S \to E) \to [0, \infty[$ is a {\bf premodifier} if
    $$h_{\Lambda_2}(\zeta)h_{\Lambda_1}(\eta) = h_{\Lambda_1}(\zeta)h_{\Lambda_2}(\eta)$$
    for all $\Lambda_1 \subseteq \Lambda_2$ and all $\zeta, \eta : S \to E$ such that $\zeta_{\Lambda_1^c} = \eta_{\Lambda_1^c}$.
\end{definition}

% TODO: Probably split into several items
% We don't need this. This is just for fun.
\begin{lemma}[Modifiers are premodifiers]
    \label{lem:modif-premodif}
    \uses{def:modif, def:premodif}
    % \lean{}
    % \leanok
    If $\rho$ is a modifier of $\ISSSD(\nu^S)$, then it is a premodifier if any of the following conditions hold:
    \begin{enumerate}
        \item $E$ is countable and $\nu$ is equivalent to the counting measure.
        \item $E$ is a second countable Borel space.
        \item $\nu$ is everywhere dense.
        \item For all $\Lambda_1 \subseteq \Lambda_2$ and all $\eta : S \to E$, $\zeta \mapsto \rho_{\Lambda_1}(\zeta \eta_{\Lambda_1^c})$ is continuous on $E^{\Lambda_1}$.
    \end{enumerate}
\end{lemma}
\begin{proof}
    \uses{prop:modif-char}
    % \leanok

    \begin{enumerate}
        \item Use Proposition \ref{prop:modif-tfae}.
        \item Omitted.
        \item Omitted.
        \item Omitted.
    \end{enumerate}
\end{proof}

\begin{lemma}[Premodifiers give rise to modifiers, Remark 1.32]
    \label{lem:premodif-modif}
    \uses{def:modif, def:premodif}
    \lean{Specification.IsPremodifier.isModifier_div}
    \leanok
    If $h$ is a premodifier and $\nu$ is such that $0 < \nu_\Lambda h_\Lambda < \infty$ for all $\Lambda$, then
    $$\rho_\Lambda := \frac{h_\Lambda}{\ISSSD(\nu)_\Lambda h_\Lambda}$$
    is a modifier of $\ISSSD(\nu)$.
\end{lemma}
\begin{proof}
    \uses{prop:modif-indep-char}

    TODO
\end{proof}

% Skipping Theorem 1.33 for now
